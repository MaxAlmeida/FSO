\chapter{Descrição de Parâmetros}
Este capitulo descreve os parametros utilizados nas funções relativas à IPC
\section{Fila de Mensagens}
Para criação de fila de mensagens é utilizado o seguinte método:
\begin{lstlisting}
msgget(IPC_PRIVATE, MSG_PRM | IPC_CREAT | IPC_EXCL);
\end{lstlisting}

O método msgget() é reponsável pela criação de fila de mensagens caso elas não exista , caso existam ele irá conectar a fila de mensagem existente. Os parametros utilizados nesse caso serão descritos a seguir:

\begin{itemize}
	\item IPC\_PRIVATE - É um key type específico para criação de uma nova fila de mensagem;
    \item MSG\_PRM - Definição da permissão da fila de mensagem, no código esse valor ficou definido como 600;
	\item IPC\_CREATE e IPC\_EXCL - quando utilizadas em conjunto se a fila de mensagem já existir a função retornará EEXIST.  
\end{itemize}