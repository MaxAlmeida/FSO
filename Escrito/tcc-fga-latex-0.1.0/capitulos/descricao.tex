chapter{Descrição de Parâmetros}
Este capitulo descreve os parametros utilizados nas funções relativas à IPC
\section{Fila de Mensagens}
Para criação de fila de mensagens é utilizado o seguinte método:
\begin{lstlisting}
msgget(IPC_PRIVATE, MSG_PRM | IPC_CREAT | IPC_EXCL);
\end{lstlisting}

O método msgget() é reponsável pela criação de fila de mensagens caso elas não exista , caso existam ele irá conectar a fila de mensagem existente. Os parametros utilizados nesse caso serão descritos a seguir:

\begin{itemize}
	\item IPC\_PRIVATE - É um key type específico para criação de uma nova fila de mensagem;
    \item MSG\_PRM - Definição da permissão da fila de mensagem, no código esse valor ficou definido como 600;
	\item IPC\_CREATE e IPC\_EXCL - quando utilizadas em conjunto se a fila de mensagem já existir a função retornará EEXIST.  
\end{itemize}

Para a enviar mensagem na fila é utilizando o seguinte método:
\begin{lstlisting}
	msgsnd(qid, msg, sizeof(msg->text),0);
\end{lstlisting}

O método msgsnd é responsável em colocar uma mensagem na fila, caso esta já esteja criada. São utilizados os seguintes paramentros:

\begin{itemize}
	\item qid - É o parâmetro que recebe como valor o id da fila de mensagem criada.
	\item msg - É a mensagem que será enviada na fila pelo pai.
	\item sizeof(msg->text) - É o tamanho que o texto da mensagem terá.
\end{itemize}

Para que a mensagem que foi enviada na fila seja lida, usa-se o seguinte método:
\begin{lstlisting}
	msgrcv(qid, msg, sizeof(msg->text),0,0);
\end{lstlisting}

O método  msgrcv é responsável por receber e ler a mensagem que foi enviada para ele na fila de mensagem. Alguns parametros são usados e são eles:

\begin{itemize}
	\item qid - É o parâmetro que recebe como valor o id da fila de mensagem recebida.
	\item msg - É a mensagem que será enviada na fila pelo pai.
	\item sizeof(msg->text) - É o tamanho que o texto da mensagem terá.
\end{itemize}


Para que haja um controle das operações que são realizadas na fila é usado o seguinte método:
\begin{lstlisting}
	msgctl(qid, IPC_RMID, NULL);
\end{lstlisting}

Esse método msgctl executa a operação de controle na fila de mensagem. Os parametros que são usados, são os seguintes:
\begin{itemize}
	\item qid - É o parâmetro que recebe como valor o id da fila de mensagem enviadas e  recebidas.
	\item IPC\_RMID - É o parametro que marca o segmento a ser destruido. Tal segmento só é realmente destruido após o último processo que é desanexado.
	\item NULL - Retorna nulo para o método.

\end{itemize}

\section{Memória Compartilhada}
O Método utilizado para criação de memória compartilhada é utilizado msget():
	\begin{lstlisting}
		shmget(key, SHM_LEN, IPC_CREAT | 0666);
	\end{lstlisting}
	
A seguir uma descrição dos parâmetros utilizados:
\begin{itemize}
  \item key - é o valor utilizado para acessar a memória;
  \item SHM\_LEN - é o tamanho em bytes do tamanho da memória a ser criado. No nosso caso o valor de SHM\_LEN é 250; 
  \item IPC\_CREATE | 0666 - Flag de criação e permissão de acesso;
\end{itemize}	

O shmat() é utlizado para dá o attach no segmento de memória compartilhada. Retorna um ponteiro para o segmento de memória compartilhada.
\begin{lstlisting}
	shmat(shmid, NULL, 0);
\end{lstlisting}

\begin{itemize}
\item shmid - É  id da memória criada 
\item NULL - Quando NULL deixa para o sistema a escolha do endereço de memória utilizada.
\item No terceiro parametro indica se a mensagem será apenas lida para isso deveria ser passada a flag  SHM\_RDONLY , não é nosso caso então passa-se a o valor 0.
\end{itemize}

shmctl() é utlizado para alterar características do segmento de memória compartilhada.
\begin{lstlisting}
	shmctl(shmid, IPC_RMID, NULL);
\end{lstlisting}  	

\begin{itemize}
	\item shmid = É o id da memórica criada
	\item IPC\_RMID = É flag utilizada para deletar a memória compartilhada
	\item O terceiro argumento é um ponteiro para a estrutura shmid\_ds , utilizada para obtenção ou alteração de características do segmento da memória. Neste caso o segmento de memória está sendo apagado, portanto não há necessidade de se passar essa estrutura.
\end{itemize}

\section{Socket TCP}
A seguir sao apresentados os metodos utilizados para comunicaçao tcp;

O método socket() é utilizado para a criação de socket. Neste trabalho este método é utilizado da seguinte maneira:
\begin{lstlisting}
	socket(AF_INET, SOCK_STREAM,0);
\end{lstlisting}

\begin{itemize}
	\item AF\_INET - Protocolo de internet utilizado , a opção AF\_INET indica que o protocolo utilizado será o IP. Uma outra opção seria utilizar o protocolo IPV6 passando o valor AF\_INET6
	\item SOCK\_STREAM - Indica o protocolo utilizado na camada de transporte , a opção SOCK\_STREAM possbilita que seja utilizado o protocolo tcp. Caso a opção escolhida fosse SOCK\_DGRAM o protocolo utilizado seria UDP.
	\item O terceiro argumento é utlizado para indicar o uso de mais protocolos. O que não é o caso deste trabalho portanto o valor configurado é 0.
\end{itemize}


O método bind() é utilizado para atribuir a porta e o endereço ao qual socket irá utilizar.

\begin{lstlisting}
	bind_port( int socket,struct sockaddr_in server );
\end{lstlisting}	

\begin{itemize}
	\item int socket - Identificar do socket criado pelo método socket();
	\item sockaddr\_in server - Estrutura de dados utilizada para passar informações como endereço ip e porta a ser utilizada;  
\end{itemize}

O método connect() é utilizado no lado client para conectar ao servidor.

\begin{lstlisting}
	connect(socket,(struct sockaddr *)&server, sizeof(server));
\end{lstlisting}

\begin{itemize}
	\item socket - Valor do socket criado para estabelecimento da conexão;
	\item (struct sockaddr *)\&server - Estrutura com informações referentes a porta e ip do servidor;
	\item  sizeof(server) - Tamanho em bytes da variável server referente ao servidor.
\end{itemize}
