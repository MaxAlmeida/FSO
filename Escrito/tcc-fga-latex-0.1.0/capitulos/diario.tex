\chapter{Diario de Atividades}
Essa seção descreve em topicos as atividades efetuadas bem como suas respectivas datas.

O trabalho deu ínicio no dia 28 de março, onde os alunos fizeram pesquisas sobre filas de mensagens, no qual eles encontraram os respectivos tutorias que estão listados abaixo e replicaram os exemplos a fim de obter conhecimento sobre o assunto.

\begin{itemize}
	\item http://www.ime.uerj.br/~alexszt/cursos/so1/troca\%20de\%20mensagens.pdf
	\item http://beej.us/guide/bgipc/output/html/singlepage/bgipc.html\#mq
\end{itemize}

No dia 31 de março as comunicação entre o pai A com o filho A e o filho B com o Pai B ja haviam sido implementadas, contudo faltava a comunicação desses dois hosts através do filho a com o filho B pela memória compartilhada. Neste momento, a equipe tinha bastante dificuldade com relação a memória compartilhada, principalmente na recuperação  da mensagem pela memória o que ocasionou um certo atraso na comunicação unidirecional. Também fora estudado o seguinte material para melhor esclarecimento sobre a implementação  de memoria compartilhada:

\begin{itemize}
	\item https://www.cs.cf.ac.uk/Dave/C/node27.html

\end{itemize}
	
No dia 8 iniciou-se o estudo de implementações voltadas para comunicação via tcp a partir dos seguintes materiais:

\begin{itemize}
	\item http://www.theinsanetechie.in/2014/01/a-simple-chat-program-in-c-tcp.html
	\item http://www.programminglogic.com/example-of-client-server-program-in-c-using-sockets-and-tcp/
	\item http://www.binarytides.com/socket-programming-c-linux-tutorial/	
\end{itemize}

No dia 9 de abril foram iniciado a implementação da comunicação via tcp em um dos hosts e, por fim, no dia 10 de abril fora concluído o trabalho com a comunicação duplex via tcp entre os hosts. 

\section{Problemas Encontrados}
Uma das dificuldades da equipe foi em desenvolver uma solução que possibilitasse que o filho do processo ‘A’ só escrevesse na memória compartilhada e , consequentemente, só pegasse a nova mensagem na fila, quando a mesma já tivesse sido lida pelo filho  do Processo B (a).

Na comunicação duplex teve-se dificuldade no envio e recebimento de mensagens na mesma fila de mensagens. O que acarretava em uma leitura equivocada por parte dos processos pai das mensagens que estavam sendo recebidas e enviadas (b).

\section{Soluçoes Adotadas}
Para o problema(a) uma abordagem inicial consistia na remoção da memoria compartilhada, de modo que o filho A utilizasse o código de erro para a criação de uma nova área de memória, e pudesse pegar a mensagem na fila fornecida pelo pai A.  A segunda abordagem foi o uso de sinais para a comunicação entre os processo filho A e filho B , de modo que o filho a identificasse quando a leitura feita por B fosse terminada.Entretanto, nenhuma dessas abordagens a equipe obteve êxito, tendo-se muita dificuldade em implementa las. A abordagem que se obteve êxito consiste no uso de um código escrito na área da memória de modo que possibilite que o filho a reconhece quando se deve escrever na memória. Assim, após a leitura da mensagem enviada pelo filho a pela memória, o filho ‘B’ escreve um código ( ex: 8800) na memória, o filho ‘A’ ao ler esse código identificar que pode ser feita escrita na memória , sendo , consequentemente , lida uma nova mensagem na fila disponibilizada pelo pai ‘A’.

Para o problema(b) a solução encontrada para a comunicação duplex, foi a criação de duas filas de mensagens. Uma para envio de mensagens e outra para recebimento de mensagens por parte dos processos pai.


 	